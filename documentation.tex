\documentclass[12pt, dvipsnames, svgnames, x11names, oneside]{book}

% URLs and hyperlinks ---------------------------------------
\usepackage{hyperref}
\hypersetup{
	colorlinks=true,
	linkcolor=blue,
	filecolor=magenta,      
	urlcolor=blue,
	pdftitle=Plann,
}
\usepackage{xurl}
%-------------------------------------------------

% inline lists ----------------------------------------------
\usepackage[inline]{enumitem}
% -----------------------------------------------------------

% environments ----------------------------------------------
\newenvironment{sansserif}{\sffamily}{\normalfont}
%------------------------------------------------------------

% logic gates -----------------------------------------------
\usepackage{circuitikz}
\usetikzlibrary{calc}
%------------------------------------------------------------

\usepackage[stable]{footmisc}
\usepackage{float}
\usepackage{rotating}
\usepackage{graphicx}
\usepackage{xcolor}

% MIPS Assembly language definition for the LaTeX `listings' package
%
% The list of instructions and directives are those understood by the
% MARS MIPS Simulator [http://courses.missouristate.edu/KenVollmar/MARS/]
%
% Author: Eric Walkingshaw <eric@walkingshaw.net>
%
% This code is in the public domain.
%
% Here is an example style. I like it for slides, but you might want 
% something a bit less noisy for print. :)
%
% \definecolor{CommentGreen}{rgb}{0,.6,0}
% \lstset{
	%   language=[mips]Assembler,
	%   escapechar=@, % include LaTeX code between `@' characters
	%   keepspaces,   % needed to preserve spacing with lstinline
	%   basicstyle=\footnotesize\ttfamily\bfseries,
	%   commentstyle=\color{CommentGreen},
	%   stringstyle=\color{cyan},
	%   showstringspaces=false,
	%   keywordstyle=[1]\color{blue},    % instructions
	%   keywordstyle=[2]\color{magenta}, % directives
	%   keywordstyle=[3]\color{red},     % registers
	% }

\ProvidesPackage{mips}

\RequirePackage{listings}

\lstdefinelanguage[mips]{Assembler}{%
	% so listings can detect directives and register names
	alsoletter={.\$},
	% strings, characters, and comments
	morestring=[b]",
	morestring=[b]',
	morecomment=[l]\#,
	% instructions
	morekeywords={[1]abs,abs.d,abs.s,add,add.d,add.s,addi,addiu,addu,%
		and,andi,b,bc1f,bc1t,beq,beqz,bge,bgeu,bgez,bgezal,bgt,bgtu,%
		bgtz,ble,bleu,blez,blt,bltu,bltz,bltzal,bne,bnez,break,c.eq.d,%
		c.eq.s,c.le.d,c.le.s,c.lt.d,c.lt.s,ceil.w.d,ceil.w.s,clo,clz,%
		cvt.d.s,cvt.d.w,cvt.s.d,cvt.s.w,cvt.w.d,cvt.w.s,div,div.d,div.s,%
		divu,eret,floor.w.d,floor.w.s,j,jal,jalr,jr,l.d,l.s,la,lb,lbu,%
		ld,ldc1,lh,lhu,li,ll,lui,lw,lwc1,lwl,lwr,madd,maddu,mfc0,mfc1,%
		mfc1.d,mfhi,mflo,mov.d,mov.s,move,movf,movf.d,movf.s,movn,movn.d,%
		movn.s,movt,movt.d,movt.s,movz,movz.d,movz.s,msub,msubu,mtc0,mtc1,%
		mtc1.d,mthi,mtlo,mul,mul.d,mul.s,mulo,mulou,mult,multu,mulu,neg,%
		neg.d,neg.s,negu,nop,nor,not,or,ori,rem,remu,rol,ror,round.w.d,%
		round.w.s,s.d,s.s,sb,sc,sd,sdc1,seq,sge,sgeu,sgt,sgtu,sh,sle,%
		sleu,sll,sllv,slt,slti,sltiu,sltu,sne,sqrt.d,sqrt.s,sra,srav,srl,%
		srlv,sub,sub.d,sub.s,subi,subiu,subu,sw,swc1,swl,swr,syscall,teq,%
		teqi,tge,tgei,tgeiu,tgeu,tlt,tlti,tltiu,tltu,tne,tnei,trunc.w.d,%
		trunc.w.s,ulh,ulhu,ulw,ush,usw,xor,xori},
	% assembler directives
	morekeywords={[2].align,.ascii,.asciiz,.byte,.data,.double,.extern,%
		.float,.globl,.half,.kdata,.ktext,.set,.space,.text,.word},
	% register names
	morekeywords={[3]\$0,\$1,\$2,\$3,\$4,\$5,\$6,\$7,\$8,\$9,\$10,\$11,%
		\$12,\$13,\$14,\$15,\$16,\$17,\$18,\$19,\$20,\$21,\$22,\$23,\$24,%
		\$25,\$26,\$27,\$28,\$29,\$30,\$31,%
		\$zero,\$at,\$v0,\$v1,\$a0,\$a1,\$a2,\$a3,\$t0,\$t1,\$t2,\$t3,\$t4,
		\$t5,\$t6,\$t7,\$s0,\$s1,\$s2,\$s3,\$s4,\$s5,\$s6,\$s7,\$t8,\$t9,%
		\$k0,\$k1,\$gp,\$sp,\$fp,\$ra},
}[strings,comments,keywords]

\definecolor{CommentGreen}{rgb}{0,.6,0}
\lstset{frame=tb,
		aboveskip=3mm,
		belowskip=3mm,
	   language=[mips]Assembler,
	   escapechar=@, % include LaTeX code between `@' characters
	   basicstyle=\footnotesize\ttfamily\bfseries,
	   commentstyle=\color{CommentGreen},
	   stringstyle=\color{cyan},
	   showstringspaces=false,
	   keywordstyle=[1]\color{blue},    % instructions
	   keywordstyle=[2]\color{magenta}, % directives
	   keywordstyle=[3]\color{red},     % registers
	 }

\title{Documentation of CPU project}
\author{
	Mahdi Haghverdi \\
	Seyed Hussein Husseini \\
	Sina Rabiei
}

\renewcommand{\arraystretch}{1.25}

% section numbering -----------------------------------------
\setcounter{secnumdepth}{3}
\setcounter{tocdepth}{3}
%------------------------------------------------------------

\begin{document}
	\maketitle
	\frontmatter
	\tableofcontents
%	\chapter{Preface}
	\mainmatter
	
	\chapter{ISA}\label{ch:isa}
		\begin{sansserif}
			In this chapter we will introduce the ISA of our CPU. Many different and important aspects of the ISA will be covered. Covered topics are:
            \begin{itemize}
				\item \nameref{sec:isa-table}
				\item \nameref{sec:isa-insts}
				\item \nameref{sec:isa-encode}
			\end{itemize}
		\end{sansserif}
				
		\section{ISA Table}\label{sec:isa-table}
			The given ISA to our group consists of 12 commands, spread into arithmetic, logical, branching and memory categories.
			
			\begin{table}[hb]
				\caption{ISA table}
				\begin{center}
					\begin{tabular}{|c|c|}
						\hline
						Type & Assembly Instruction \\
						\hline
						\hline
						Arithmetic & \texttt{add, sub, mul} \\
						\hline
						Logical & \texttt{and, or, nor, nand, andi, ori} \\
						\hline
						Branching & \texttt{bnq} \\
						\hline
						Memory & \texttt{lw, sw} \\
						\hline
					\end{tabular}
				\end{center}
			\end{table}\label{sec:table:isa-table}
		
				Do note that in our CPU all ISA instructions mentioned in table \ref{sec:table:isa-table}, deal with registers to get and store data; no operation can be dealt with direct access to memory.
		
		\section{ISA Instruction Formats}\label{sec:isa-insts}
		In this CPU we have two types of instruction formats: 
		\begin{enumerate*}
			\item \texttt{R-type}\footnote{Register format instruction}
			\item \texttt{V-type}\footnote{Value format instruction}
		\end{enumerate*}; and in this section we will go through details of this formats. \\
		
		These are the the divided assembly instructions of the CPU:
		\begin{table}[h]
			\caption{\texttt{r-type} and \texttt{v-type} instructions}
			\begin{center}
				\begin{tabular}{|c|c|c|}
					\hline
					Type & Assembly Instruction & Count \\
					\hline
					\hline
					\texttt{R-type} & \texttt{add, sub, mul, and, or, nor, nand} & 7 \\
					\hline
					\texttt{V-type} & \texttt{bnq, lw, sw, andi, ori} & 5 \\
					\hline
				\end{tabular}
			\end{center}
		\end{table}\label{sec:table:isa-insts}
		
		The instruction format for \textbf{recognizing the instruction}\footnote{encoding the assembly instruction in binary}, looks like this:
		
		\begin{table}[h]
			\caption{Instruction format of \texttt{r-type} and \texttt{v-type} instructions}
			\begin{center}
				\begin{tabular}{|c|c|c|c|c|c|c|}
					\hline
					Type & Opcode & Rs & Rt & Rd & Don't Care & Funct \\
					\hline
					\texttt{R} & \texttt{000} & \texttt{nnnn} & \texttt{nnnn} & \texttt{nnnn} & \texttt{14-bit} & \texttt{nnn} \\
					\hline
					\hline
					Type & Opcode & Rs & Rt & \multicolumn{3}{|c|}{immediate value} \\ \hline
					\texttt{V} & \texttt{nnn} & \texttt{nnnn} & \texttt{nnnn} & \multicolumn{3}{|c|}{\texttt{21-bit}} \\
					\hline
				\end{tabular}
			\end{center}
		\end{table}
		
		\section{ISA Encoding}\label{sec:isa-encode}
		All CPUs only understand zeros and ones, in order to tell the CPU to do this and to do that, we have to encode the instructions.
		
		The instruction code table is below:
		
		\begin{table}[h]\label{sec:table:isd-encode}
			\caption{ISA encoding}
			\begin{center}
				\begin{tabular}{|c|c|c|}
					\hline
					Instruction (\texttt{r-type}) & Opcode & Funct \\
					\hline
					\texttt{add} & \texttt{000} & \texttt{001} \\
					\hline
					\texttt{sub} & \texttt{000} & \texttt{010} \\
					\hline
					\texttt{mul} & \texttt{000} & \texttt{011} \\
					\hline
					\texttt{and} & \texttt{000} & \texttt{100} \\
					\hline
					\texttt{or} & \texttt{000} & \texttt{101} \\
					\hline
					\texttt{nor} & \texttt{000} & \texttt{110} \\
					\hline
					\texttt{nand} & \texttt{000} & \texttt{111} \\
					\hline		
					\hline
					Instruction (\texttt{v-type}) & \multicolumn{2}{|c|}{Opcode} \\
					\hline
					\texttt{bnq} & \multicolumn{2}{|c|}{\texttt{111}} \\
					\hline
					\texttt{lw}  & \multicolumn{2}{|c|}{\texttt{001}} \\
					\hline
					\texttt{sw} & \multicolumn{2}{|c|}{\texttt{010}} \\
					\hline
					\texttt{andi} & \multicolumn{2}{|c|}{\texttt{100}} \\
					\hline
					\texttt{ori} & \multicolumn{2}{|c|}{\texttt{101}} \\
					\hline
				\end{tabular}
			\end{center}
		\end{table}
		
		\chapter{Registers}
		\begin{sansserif}
			In this chapter we will look at the register file of the CPU and the design decisions of this important part. Covered topics are:
			\begin{itemize}
				\item \nameref{sec:num-of-regs}
				\item \nameref{sec:name-of-regs}
				\item \nameref{sec:cap-of-regs}
			\end{itemize}
		\end{sansserif} 
		
		All the operations which take place in the CPU get their operands and store their results in the register file. For example the instruction:
		
		\begin{lstlisting}
add $r1, $r2, $r3
		\end{lstlisting}
		
		\noindent Which means \texttt{\$r1 = \$r2 + \$r3}, the data are fetched from the registers \texttt{\$r1, \$r2} and \texttt{\$r3}.
		
		There are some decisions for the design of the register file of the CPU which will be discussed below.
		
		\section{Number Of Registers}\label{sec:num-of-regs}
		After discussions between the team members we decided to put 16 registers in the register file of the CPU.
		
		With this number of registers, we can access them by a four-bit number, because $16 = 2^4$
		
		\section{Naming Conventions of Registers}\label{sec:name-of-regs}
		The registers are names as follows:
		\begin{enumerate}
			\item \texttt{\$zero}
			\item \texttt{\$at}
			\item \texttt{\$a1}
			\item \texttt{\$a2}
			\item \texttt{\$a3}
			\item \texttt{\$l1}
			\item \texttt{\$l2}
			\item \texttt{\$l3}
			\item \texttt{\$t1}
			\item \texttt{\$t2}
			\item \texttt{\$t3}
			\item \texttt{\$t4}
			\item \texttt{\$t5}
			\item \texttt{\$t6}
			\item \texttt{\$t7}
			\item \texttt{\$t8}
		\end{enumerate}
		
		\subsection{Register Specifications}
		These register specifications are done just for convenience and safety of the compiled programs. For example, a logical operation operands are better to be placed in \texttt{\$l\#} registers. This convention will guarantee the data safety during the program execution.
		
		\subsubsection{\texttt{\$zero} register}
		Because the value \texttt{0} is used a lot, we put it here. This register will not be overwritten.
		\subsubsection{\texttt{\$at} register}
		This is the register provided for the assembler to use on its own.
		\subsubsection{\texttt{a\#} registers}
		\texttt{a\#} registers are named after the word \texttt{arithmetic}. These registers are specific for arithmetic operations.
		\subsubsection{\texttt{l\#} registers}
		\texttt{l\#} registers are named after the word \texttt{logical}. These registers are specific for arithmetic operations.
		
		\subsubsection{\texttt{t\#} registers}			
		\texttt{t\#} registers are named after the word \texttt{temporary}. These registers can be used freely by the compiler or assembly programmer.
		
		\section{Capacity Of Registers}\label{sec:cap-of-regs}
		This CPU's registers are 32-bit registers with the capability of parallel load and read.
		
		\chapter{Control Unit}
		\begin{sansserif}
			Control Unit and Data Path are the two hearts of all CPUs. In this chapter we will show the design of our CPU's control unit.
		\end{sansserif}
		
		Control unit, produces some signals to power on or power off some parts of the data path to control the data flow between the gates, registers, the alu and multiplexers of the data path. The table below shows all the control unit signals we need for our ISA (discussed in \nameref{ch:isa} chapter)
		Here is the table of all 
		
		\section{Control Unit Signals Table}
		
			The meaning of zeros and ones in the table, are explained in the footnote of the table.
					
		\begin{sidewaystable}
			\caption{Control Unit Signals}
			\begin{center}
				\begin{tabular}{|c|c|c|c|c|c|c|c|c|c|c|c|c|}
					\hline
					Control Signal &
					\texttt{add} &
					\texttt{sub} &
					\texttt{mul} &
					\texttt{and} &
					\texttt{or} &
					\texttt{nor} &
					\texttt{nand} &
					\texttt{bnq} &
					\texttt{lw} &
					\texttt{sw} &
					\texttt{andi} &
					\texttt{ori} \\
					\hline
					\hline
					\texttt{RegDst} & 
					\texttt{1}\footnote{\texttt{Rd}} &
					\texttt{1} &
					\texttt{1} &
					\texttt{1} &
					\texttt{1} &
					\texttt{1} &
					\texttt{1} &
					\texttt{x} &
					\texttt{0}\footnote{\texttt{Rt}} &
					\texttt{x} &
					\texttt{0} &
					\texttt{0} \\
					\hline
					\texttt{ALUSrc} &
					\texttt{0}\footnote{a register} &
					\texttt{0} &
					\texttt{0} &
					\texttt{0} &
					\texttt{0} &
					\texttt{0} &
					\texttt{0} &
					\texttt{0} &
					\texttt{1}\footnote{immediate value} &
					\texttt{1} &
					\texttt{1} &
					\texttt{1} \\ 
					\hline
					\texttt{MemToReg} &
					\texttt{0}\footnote{from ALU result to a register} &
					\texttt{0} &
					\texttt{0} &
					\texttt{0} &
					\texttt{0} &
					\texttt{0} &
					\texttt{0} &
					\texttt{x} &
					\texttt{1}\footnote{from memory to a register} &
					\texttt{x} &
					\texttt{0} &
					\texttt{0} \\ 
					\hline
					\texttt{RegWrite} &
					\texttt{1}\footnote{write to register from ALU} &
					\texttt{1} &
					\texttt{1} &
					\texttt{1} &
					\texttt{1} &
					\texttt{1} &
					\texttt{1} &
					\texttt{x} &
					\texttt{1} &
					\texttt{x} &
					\texttt{1} &
					\texttt{1} \\ 
					\hline
					\texttt{MemWrite} &
					\texttt{0} &
					\texttt{0} &
					\texttt{0} &
					\texttt{0} &
					\texttt{0} &
					\texttt{0} &
					\texttt{0} &
					\texttt{0} &
					\texttt{0} &
					\texttt{1}\footnote{write to memory from ALU} &
					\texttt{0} &
					\texttt{0} \\ 
					\hline
					\texttt{Branch} &
					\texttt{0} &
					\texttt{0} &
					\texttt{0} &
					\texttt{0} &
					\texttt{0} &
					\texttt{0} &
					\texttt{0} &
					\texttt{1}\footnote{put one in the branch's \texttt{and} gate.} &
					\texttt{0} &
					\texttt{0} &
					\texttt{0} &
					\texttt{0} \\ 
					\hline
					\texttt{ALUctrl} &
					\texttt{add} &
					\texttt{sub} &
					\texttt{mul} &
					\texttt{and} &
					\texttt{or} &
					\texttt{nor} &
					\texttt{nand} &
					\texttt{sub} &
					\texttt{add} &
					\texttt{add} &
					\texttt{and} &
					\texttt{or} \\ 
					\hline
				\end{tabular}
			\end{center}
		\end{sidewaystable}\label{sec:table:cu}
	
		% \usepackage[stable]{footmisc}
		\section{Control Unit Design\footnote{These pictures will be designed in Lagisim and replaced}}
			\begin{figure}[H]
				\begin{center}
					\includegraphics[width=15cm, height=13cm, angle=90]{./images/cu}
				\end{center}
			\caption{Main Control Unit Design}
			\end{figure}
			
			\begin{figure}[H]
				\begin{center}
					\includegraphics[width=18cm, height=13cm, angle=90]{./images/aluctrl}
				\end{center}
				\caption{ALU Control truth table}							
			\end{figure}
		
			\begin{figure}[H]
				\begin{center}
					\includegraphics[width=18cm, height=13cm, angle=90]{./images/aluctl}
				\end{center}
				\caption{ALU Control Design}							
			\end{figure}
\end{document}